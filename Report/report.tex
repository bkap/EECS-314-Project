\documentclass[10pt,letterpaper]{article}
\usepackage{palatino, graphicx, amsmath, amsfonts}
\usepackage[margin=1in]{geometry}

\title{EECS 314 Final Project Report}
\author{Jacob Emmert-Aronson, Andrew Johnson,\\Ben Kaplan, and Frankie Lutikoff}
\date{May 2, 2010}

\begin{document}
\maketitle

Our goal was to create a calculator for general-purpose scientific use in MIPS
assembly language. The calculator should be able to support the entry of complex
expressions. Furthermore, the user should be able to enter operands of double
precision and the results of the computation should also be in double precision.
In addition to the four basic operations, the calculator should support
exponentiation and roots, trigonometric functions, inverse trigonometric
functions, and natural logarithms.

The first major challenge we faced was parsing the user input. Our goal was to
support entering complex expressions on a single line of input, instead of
allowing only one expression per line. SPIM provides syscalls for reading one
double precision number from the input, but there is no way to read a sequence
of doubles mixed with strings. As such, we had no choice but to read the entire
line of input and parse it ourselves. This led to two distinct issues. Once the
string was broken up into individual operators and operands, we had to determine
which were operators and which were operands. Then the operands had to be
converted from strings to actual double precision numbers. Both of these
required sufficient error handling to determine when the user entered an illegal
expression.

The next major challenge was implementing the more complex operations, which
included trigonometry, inverse trigonometry, and natural logarithms. We had to
find a way to compute all of those operations without requiring overly complex
mathematical operations. Furthermore, those computations needed to retain enough
precision to still be useful to the user. Finally, any computations had to be
efficient and terminate relatively quickly. Especially for the inverse
trigonometric functions, some methods of computation have issues with
convergence around the edges of the domain, so those had to be avoided.

The first and most important component we realized was the string functions,
which include string comparison and \texttt{atof}, to convert from a string to a
double. Without these functions, the input parser would have been entirely
impossible and we would have been limited to one operation per line. These
functions were conceptually difficult to implement in assembly language. These
functions are called many times for each expression entered by the user, so they
must be efficient and quick. Furthermore, they are required to gracefully handle
errors, as they are the primary means for determining if the user entered an
illegal expression. To overcome these challenges, we used \texttt{atof} and
\texttt{strcmp} from the C standard library as models. As a result, we produced
an efficient, elegant solution to the problem.

The input parser was the next key component we realized. Because of the strong
\texttt{atof} and \texttt{strcmp} we produced, writing the parser was much
easier than it otherwise would have been. The key design challenge here was
choosing between infix, prefix, or postfix notation. Infix is more familiar to
the average user, while postfix is a simpler algorithm to implement. We chose to
use postfix notation. While we did work on an infix parser, it was quite
complex. As such we felt that we could implement a more efficient and reliable
calculator by using postfix notation.

The final component of the calculator was the implementations of the operators
themselves. The four basic operations were simple as we could use the built-in
floating point operations. Exponentiation and roots presented the first issue.
We could have used one operator to perform both operations, requiring the user
to enter \texttt{2 .5} \verb|^| to compute the square root of two, or two separate
operations, requiring the user to enter \texttt{2 2 \~} to compute the square
root of two. We opted for the first option. That operator first computes the
exponent as if there was no fractional part of the exponent, and then computes
the fractional part and properly combines the results. This resulted in a
simpler interface for the user.

For the trigonometric functions, inverse trig functions, and natural logarithms,
we chose to use Taylor series. Thus we could use simpler operations to compute
these more complex functions. The main design issue was the convergence of the
Taylor series. Inverse trig especially had issues with converging near the
boundaries of the domain. Also, all of those functions other than the
trigonometric functions are undefined in certain areas. To address these issues,
we implemented a bounds check where appropriate and used an iteration limit.
Therefore the series would either converge or we cut off computation after some
large number of steps. This allows these functions to compute the result in a
short period of time regardless of the input.

Each supported operation has been implemented as its own function. Some
operations, such as the trigonometric functions, use helper functions to compute
their results. Furthermore, reading and parsing user input is its own function,
as is error handling and printing the final result. Finally, \texttt{atof} and
the string comparison functions are also implemented as functions. The reading
and parsing function is run immediately upon starting the program. It determines
what is in the input by calling the string comparison functions. When it
encounters an operand, it calls \texttt{atof} and puts resulting double on the
stack. When it encounters an operator, it calls the appropriate operator
function. All operands and results are passed between functions using the stack.
That function calls helper functions as appropriate. When it is finished, it
puts the result on the stack and returns to the read/parse function. If that
function determines that the user input has been exhausted, it calls the
function to print the final result. If an error is encountered at any time, the
appropriate error function is called.

The calculator is run by calling \texttt{spim -f calc.s} from the terminal.
Alternatively, \texttt{calc.s} can be loaded into PCSpim. When the calculator is
run, no message is presented to the user. One may simply begin typing the
expression. Due to limitations in SPIM's syscalls and the parser, no expression
can be more than 500 characters and no individual operand may be more than 20
characters. Furthermore, the number stack allocates space for only sixteen
doubles at any one time, so no more than 16 operands may be entered in a row.
However, any number of operations can appear in the expression. This is a
reverse Polish calculator, so all operators follow the operands. This
table lists all operators supported by the calculator and the operations they
perform:
\begin{table}[h!]
    \centering
    \begin{tabular}[h!]{| c | c |}
        \hline
        Operand & Operation Performed\\
        \hline
        + & Addition\\
        \hline
        - & Subtraction\\
        \hline
        * & Multiplication\\
        \hline
        / & Division\\
        \hline
        \verb|^| & Exponentiation and Roots\\
        \hline
        sin & Sine\\
        \hline
        cos & Cosine\\
        \hline
        tan & Tangent\\
        \hline
        sec & Secant\\
        \hline
        csc & Cosecant\\
        \hline
        cot & Cotangent\\
        \hline
        log & Natural Logarithm\\
        \hline
        asin & Inverse Sine\\
        \hline
        acos & Inverse Cosine\\
        \hline
        atan & Inverse Tangent\\
        \hline
        asec & Inverse Secant\\
        \hline
        acsc & Inverse Cosecant\\
        \hline
        acot & Inverse Cotangent\\
        \hline
        quit & Quit the Calculator\\
        \hline
    \end{tabular}
    \caption{Supported Operations}
    \label{tab:ops}
\end{table}\\
Once finished entering the expression, press enter to begin computation. When
the calculator has computed the final result, it will be printed on the next
line. The final result is not retained for use in later computations. Now a new
expression can be entered. If an expression results in an error, such as stack
overflow or an illegal expression, a message will inform the user of the error
and reset the calculator. If an expression results in $\pm\infty$ or \texttt{NaN},
that value will be printed, not an error message.

\end{document}
